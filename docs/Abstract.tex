%%
% 摘要信息
% 摘要内容应概括地反映出本论文的主要内容,主要说明本论文的研究目的、内容、方法、成果和结论。要突出本论文的创造性成果或新见解,不要与引言相 混淆。语言力求精练、准确,以 300—500 字为宜。
% 关键词是供检索用的主题词条,应采用能覆盖论文主要内容的通用技术词条(参照相应的技术术语 标准)。按词条的外延层次排列(外延大的排在前面)。。


\cabstract{
	摘要内容应概括地反映出本论文的主要内容,主要说明本论文的研究目的、内容、方法、成果和结论。要突出本论文的创造性成果或新见解,不要与引言相混淆。语言力求精练、准确,以300—500字为宜。
	
	要开始一个新的段落,在LaTex源文件里面就是增加一个空行。
}
% 中文关键词(每个关键词之间用“;”分开,最后一个关键词不打标点符号。)
\ckeywords{研究目的;研究方法;创新性成果;独特见解 }

\eabstract{
	Image color editing is one of the most generous image processing tasks, which borrows one image’s color characteristics to another so that the color appearance of these two images are visually similar. This is a process to change image color style to another specified style. Color editing techniques can adjust the image's color and its artistic style,according to the needs of different applications, e.g. film production, photo processing and web design. The key problem is how to achieve a satisfied color editing result and preserve the contents of the source image well. 
	
	In this paper, we discover many edge-aware smooth methods and non-linear color mapping based color transfer methods in literature. Combined with geometric target region extraction and correction operation, we present two methods to achieve visually satisfied interactive edge-aware image color editing results. One is color distribution mapping based on multi-scale gradient-aware decomposition, and the other is interactive image color transfer based on multi-cue manipulation. The color distribution mapping decomposes the image editing issue into image color edge preservation and color transfer. First, input image is decomposed into multiple detail layers and base layers using edge-preserving WLS operator. 
}
% 英文文关键词(关键词之间用逗号隔开,最后一个关键词不打标点符号。)
\ekeywords{Image Editing, Edge Preserving, Color Mapping, Color Clustering, Image Inpainting}