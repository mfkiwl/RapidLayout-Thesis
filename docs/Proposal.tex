%%
% 开题报告
% 请注意由于此部分内容是在表格中实现的,因此如果需要换行的话不可采取空白行的方式,需要使用“\\”进行换行

% 选题目的
\objective{
	\\
	\ \ 近年来在FPGA大规模神经网络加速器受到来自学术界和工业界的关注,同时大型和高资源利用率的电路设计在
	FPGA上实现的困难也为EDA领域带来机遇和挑战。
	Xilinx UltraScale+ 系列FPGA上配备的的DSP48,BRAM,UltraRAM 等硬核是专为高速高性能计算设计的乘法和存储单元,
	并且支持高速级联。针对这一架构本文设计和实现了利用三种硬核与高速级联通道的
	神经网络脉动阵列加速器,包括级联的卷积计算单元和矩阵乘法运算单元。然而,对于大规模的加速器设计布局布线仍
	存在很大困难:使用Vivado布局布线大约花费5-6小时,需手动指定上千个计算存储单元的布局位置,而且无法保证
	布线后的设计可以达到时序要求。
	因此,本文希望构建一个自动的布局布线框架,将布局问题建模为
	优化问题并自动求解,寻找到令计算单元布线距离最短、时钟频率最高的最优解。然后根据最优解完成自动布局和
	站内布线,并且根据估计线长进行流水线优化,达到最高的时钟频率。
}

% 思路
\methodology{
	\\
	\ \ 第一,将布局问题建模为存在约束的多目标优化问题,因为搜索空间大,目标函数不可微分,所以使用启发式搜索算法寻找最优解\\
	\ \ 第二,使用Xilinx实验室正在开发的RapidWright项目,可以快速取得并修改RTL设计和各层级物理资源;
	从而实现物理布局布线和流水线优化,也可以将布线的线长估计作为目标函数;\\
	\ \ 第三,结合近年来在其他领域取得进展的优化算法,对进化算法在FPGA布局问题上的应用进行探索;\\
	\ \ 第四,进行基于进化算法的FPGA布局算法设计,编程实现端到端的自动FPGA布局布线框架,并与标准布局框架进行对比实验。

}

% 研究方法/程序/步骤
\researchProcedure{
	\\
	\ \ 为了能够更好地进行加速器设计与算法研究,本文遵循以下方法:
	\\
	\ \ 第一,调研近年来卷积神经网络的FPGA硬件加速设计及其经典计算架构;
	\\
	\ \ 第二,调研FPGA实现流程与经典方法,当前布局算法的种类及各自原理特性;
	\\
	\ \ 第三,调研进化算法用于FPGA布局的发展,以及当前最先进布局算法的原理及应用,对布局问题建模,使用
	先进的FPGA自定义框架结合布局算法设计编程实现完整的布局布线框架;
	\\
	\ \ 第四,与标准算法和新提出的先进布局算法进行对比实验,进一步验证其有效性,设计结果可视化方法,令结果更加直观。
}

% 相关支持条件
\supportment{\\
	\ \ 指导老师实验室长期从事FPGA基础研究以及机器学习、计算机架构设计、神经网络等哪方面的研究。实验室的计算资源
	充足,实验设备先进,确保了本文研究和实验所需要的计算资源与仪器设备。中山大学图书馆和多种网络资源可以提供本项目
	所需要的文献资料,保障了项目的顺利完成。
}

% 进度安排
\schedule{\\
	\ \ 2019.10 -- 2019.11:完成神经网络加速器的模型调研,总结和学习经典的架构设计;完成FPGA实现过程和算法的调研学习,
	总结常用算法与最先进算法,并梳理进化算法在FPGA布局问题上的应用和发展过程;\\
	\ \ 2019.11 -- 2020.01:完成神经网络脉动阵列加速器的RTL设计,对硬核问题布局建模,并学习RapidWright实现框架的使用方法;\\
	\ \ 2020.01 -- 2020.03:完成进化布局算法设计与测试,结合RapidWright搭建端到端的布局布线框架,解决物理实现问题;\\
	\ \ 2020.03 -- 2020.04:对算法进行敏感度分析与参数优化,选取对比算法设计实验,完善实验细节,并设计直观的结果可视化方法。
	
}

% 指导教师意见
\proposalInstructions{
	同意开题。
}

