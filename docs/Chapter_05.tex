\chapter{总结与展望}

本章首先进行了本文工作的总结,包括本文的创新点以及研究意义。然后分析了本文提出
算法与框架的不足之处,最后根据当前学术领域的发展趋势展望未来工作。

\section{工作总结}

随着深度神经网络的发展以及后模摩尔时代的到来,芯片领域的研究兴趣逐渐从计算密度的提高向计算架构的创新转变。
因此,领域定制架构(DSA)和基于FPGA的计算架构设计得到了充分发展。而在神经网络加速器的规模逐渐增大,
计算速度不断提高的过程中,EDA领域的问题和挑战重新回到了研究者的视线之内。以大规模脉动阵列的FPGA布局为例,
由于其较高的资源利用率与目前异构FPGA结构的不规则性,无法使用现有的布局工具得到可布线的结果,只能依靠
手工设计。针对这一问题本文提出RapidLayout框架,基于进化算法解决布局难题,并加速FPGA设计进程。

全文的主要工作以及贡献点如下:


\begin{enumerate}
    \item 本文将异构FPGA的硬核布局建模为多目标优化问题,应用NSGA-II与CMA-ES两种进化算法求解,在运行时间和结果质量的多个方面超越state-of-the-art布局方法。
    \item 本文针对脉动阵列的级联规则造成的布局限制,提出一种创新的基因型设计,将布局问题划分为三个子问题并将布局限制编码进优化过程;    
    \item 本文设计和实现了自动化、端到端的快速FPGA布局布线框架,RapidLayout,大大加速了FPGA的设计进程;
    \item 本文提出了布局的迁移学习方法,显著加速了硬核布局结果跨设备优化过程。
\end{enumerate}



\section{研究展望}

本文的布局框架虽在UltraScale+全系列FPGA上验证了通用性,但对其他类型加速器的支持尚待开发,因此可从以下方向进行优化:

\begin{enumerate}
    \item 增加RapidLayout对布局网表的通用性,以连通图的方式分析输入网表结构以支持任意FPGA设计;
    \item 加入对卷积计算单元和矩阵计算单元的混合布局优化;
    \item 研究细粒度布局,进一步优化全局布局结果,使布线后的时钟频率更稳定。
\end{enumerate}