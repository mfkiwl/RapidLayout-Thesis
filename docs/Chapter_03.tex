\chapter{基于进化算法的FPGA脉动阵列快速硬核布局方法}



\section{本章小结}
















































% \chapter{几何驱动的用户目标区域提取与矫正方法}
% 内容概括 \cite{zhang98}。

% \section{勾画式用户目标区域标注}
% 勾画式用户标注,是一种简单易行的标注方法\cite{hariharan14,Li08,Su81,Liu93,jiang99} 。……

% \section{基于颜色聚类的目标区域提取方法}
% \label{sec1}
% 这里的颜色分类其实是为图像目标区域提取服务的。通过对图像颜色进行分类,结合用户的标注指定,我们得到用户期望的目标区域的颜色分类,根据这些分类就能够提取出颜色传递的目标区域。……

% \section{几何驱动的目标区域边界矫正方法}
% \ref{sec1} 节提出的目标区域提取方法可以在均匀性或一致性的前提下将图像目标物体或目标区域分割出来,若与相邻部分合并则会破坏这种一致性。

% \section{几何驱动的目标区域提取与矫正实验结果分析}
% 我们进行了图像目标区域提取与矫正实验。……

% \section{本章小结}
% 本章阐述了图像局部颜色编辑方法中图像目标区域提取的相关方法,……
