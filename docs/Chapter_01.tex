\chapter{引言}
% 引言是论文正文的开端,应包括毕业论文选题的背景、目的和意义;对国内外研究现状和相关领域中已有的研究成果的简要评述;介绍本项研究工作研究设想、研究方法或实验设计、理论依据或实验基础;涉及范围和预期结果等。要求言简意赅,注意不要与摘要雷同或成为摘要的注解。

本章首先介绍FPGA在人工智能发展背景下的应用场景,大规模高计算密度的设计对FPGA实现过程带来的挑战,由此说明FPGA布局的选题背景,然后总结了
已有的FPGA布局方法和研究现状,接着介绍本文的主要工作,最后概述本文的结构与章节安排。

\section{选题背景与意义}

% 脉动阵列加速器
英特尔的创始人之一戈登$\cdot$摩尔提出,每隔18-24个月集成电路可容纳的元器件就会翻一倍,其性能也将提升一倍,这条著名论断被人们称为摩尔定律(Moore's Law)。
然而随着晶体管逐渐变小,其静态功耗不断增加,功耗上升和能量效率降低使得摩尔定律逐渐失效。在“后摩尔时代”,研究者和从业者们不再单纯
追求计算密度的提高,转而寻求计算架构的创新。相对于CPU的复杂、低并行度,与GPU的通用、高并行度,FPGA为芯片设计者们提供了一种
领域定制架构(Domain-Specific Architecture)的全新思路:普通FPGA虽然频率较低,但具有高并行、可配置、低功耗、高能量效率的特点。
对比CPU集中和复杂的内存管理方式,FPGA的存储为分布式而且简单可配置。同时,现代FPGA多配备大量专用计算单元,
可提供较大规模的计算并行度。
现代的高性能FPGA在保留其灵活性的同时可以实现近似ASIC的计算性能与时钟频率,这得益于其异构计算资源的设计:除了数百万传统的查找表(LUTs)和寄存器(Flip-Flops),异构
FPGA还在片上引入了DSP,RAM等硬核IP实现近存储计算(Near-Storage Computation),这样的硬核具有计算效率高、存储容量大、时钟频率高、支持级联等特点。同时,对可级联的硬核之间
设计了专用高速布线资源,可构建高速数据通道方便数据搬运。然而,数量庞大的计算资源、分布和尺寸不规则的异构计算存储IP、逐渐增大的计算架构规模,都为
FPGA设计的实现带来了挑战。
以Xilinx Vivado Design Suite 在 UltraScale 与 UltraScale+ 系列FPGA上实现卷积神经网络加速器为例,完整的
实现流程需要耗费数小时甚至数天,并且无法保证实现结果可以达到目标时序要求。因此,面向大型异构高计算密度FPGA的布局布线算法成为了近年来
许多学者的研究重点。

% 布局方法
FPGA设计的实现一般包括6个步骤:合成(Synthesis),工艺映射(Technology Mapping),打包(Clustering/Packing),布局(Placement),
布线(Routing),生成比特流(Generate Bit-Stream)。FPGA布局是把打包后的网表对应到板上的物理元件位置上,布局的结果直接影响设计
能否成功布线与设计的时序表现,而在实现的步骤中布局又是耗时较大的一步。因此,对FPGA布局的优化是提高CAD工具性能的关键。
已有的应用布局方法主要分为三类,基于模拟退火(Simulated Annealing)的布局算法、解析布局算法(Analytical Placement)、和Min-Cut法。
模拟退火法出现较早且应用广泛,经过随机初始化的布局在温度函数的控制下逐渐降低接受随机变化的概率,使系统冷却,从而得到最优布局结果。
退火法得到的布局结果质量高,但所需迭代优化时间较长,并行度和可扩展性较差。Min-Cut法将逻辑网表不断按区域划分,直到网表内元件完全展开
和FPGA物理位置对应,适用于小型布局问题,且局部优化容易收敛至局部最优。近年来获得较大发展的方法是解析布局法,其将FPGA布局优化问题建模为
方程系统,并使用求解算法解得布局结果。解析布局算法可扩展性强,结果质量较高,但需要代价较大的布局结果合法化过程,可并行度较差。

近年强化学习和神经网络架构搜索算法的发展促进了进化算法的兴起,诸如经典的多目标优化方法NSGA-II(Non-Dominated Sorting Genetic Algorithm, 
非支配排序遗传算法)和 State-of-the-Art 非求导优化算法CMA-ES(Covariance Matrix Adapted Evolution Strategy, 协方差矩阵适应进化策略)。
这些进化算法在许多多目标、黑箱、非平滑或非连续域优化问题上取得了领先的表现,而且进化算法具有良好的可拓展性,
其基于种群进化的优化方式天然具备适应并行计算的优势。因此,将新的进化算法应用于FPGA布局优化问题有希望解决大型异构FPGA的布局困难问题,
具有极大的可行性与研究意义。


\section{国内外研究现状和相关工作}
\label{sec:related_work}
近年来,在FPGA上实现神经网络加速器受到了研究者和从业者的广泛关注,
随着高性能、低延迟的FPGA神经网络加速器研究兴起,大规模的FPGA设计实现对
EDA工具提出了更高的要求,吸引了国内外学者的研究兴趣。本节通过对
几种高性能FPGA脉动阵列加速器的总结和基于进化算法的FPGA布局方法发展从整体
上阐述人工智能硬件加速与EDA领域的研究现状和相关工作。


\subsection{基于FPGA的脉动阵列加速器}

脉动阵列的提出解决了硬件加速器随着规模增大速度降低的问题。脉动阵列的结构是将高速计算单元以2D的方式排列并
级联形成网络,数据从一端按节拍连续输入,经过运算的结果则从另一端按照节拍连续输出。
Google TPU 首先将脉动阵列式的矩阵乘法加速器用于机器学习计算的加速。脉动阵列不仅限于
矩阵乘法,也可以用于卷积神经网络的计算加速,如CLP,Cascades,和 Xilinx SuperTile。

然而实现高利用率的设计同时保持高运算速度十分具有挑战性。CLP基于Virtex-7系列FPGA实现,
且没有使用片上DSP计算资源,其时钟频率仅为100-170 MHz。Xilinx SuperTile 在
UltraScale+ FPGA上实现,时钟频率可以达到750 MHz,但与此同时放弃了URAM的带宽,而且
浪费了50\%的片上DSP资源。Cascades 同样基于UltraScale+ FPGA 实现,其时钟频率达到
650 MHz,DSP、片上RAM的利用率均达到95\%,而且没有限制URAM的带宽。但是,高利用率
的脉动阵列设计带来了布局上的困难:Cascade设计无法依赖Vivado内置的布局引擎产生
可布线的结果,而必须人工设计480个卷积单元内部硬核的布局位置,才可以成功布局,并达到URAM上限的650 MHz
时钟频率。

\subsection{基于进化算法的FPGA布局方法}

将进化算法应用到FPGA布局优化曾有过许多尝试,但很少获得成功。最早的尝试由 Venkatraman 与 Patnaik 提出,
这项工作将FPGA上的元件位置二维坐标直接编码为遗传算法的基因型,并且使用基于关键路径长度估计和空间利用率的
目标函数进行优化。这项工作为FPGA布局方法提供了新思路,但并未取得比传统的模拟退火法更优的布局结果。
近年来,P. Jamieson 指出遗传算法无法取得与模拟退火布局算法相当的结果质量是由于crossover算子在
探索解空间的能力上有限。因此提出了Supergenes作为一种改进的遗传算法。这项工作引入了分组算子克服
crossover算子的缺点,并产生了和退火相当质量的布局结果。Praveen T. 对遗传算法进行了平行化,并将
遗传算法和退火布局方法结合,利用退火法提高布局结果的质量,用平行化的遗传算法对布局搜索过程加速。

总体而言,将进化算法应用于FPGA布局经历过较多尝试,但并未获得相比模拟退火法更优的性能。然而,近年来
强化学习和神经网络架构搜索的发展推动了新的进化算法发展:NSGA-II在强化学习的解空间探索上展示了良好性能,
CMA-ES作为一种无须求导的优化方法在架构搜索上得到了应用。并且进化算法基于种群的优化思想具有对平行计算加速
的良好支持,且不需要高代价的解合法化过程。因此,将更先进的进化算法应用于FPGA布局问题具有较大研究意义。



\section{本文的研究内容与主要工作}

本文的研究内容是通过进化算法对高计算密度的卷积神经网络加速器进行布局优化,并设计全自动、端到端的
FPGA布局布线实现框架。
% 本文提出的实现框架从以下几个方面解决大规模脉动阵列加速器硬核布局无法避免手工设计的问题:
% (1) 快速且自动发现针对特定设计与FPGA架构的正确布局方法,无需启动Vivado进行试错;
% (2) 将脉动阵列的级联规则造成的布局限制编码进自动搜索方法,避免产生不合法布局结果;
% (3) 提供快速的布局线长估计和可行解的目标函数计算;
% (4) 探索和利用硬核分布的规律进行布局结果重用,并支持”迁移学习“加快布局在不同设备上的配置。
由于高计算密度的神经网络脉动阵列加速器无法依靠现有的布局算法得到可行的全局布局结果,本文提出
对此类加速器设计进行硬核布局优化,并探索通过布局结果的复用大幅降低整体设计实现时间的方法。

本文的工作是将FPGA脉动阵列硬核布局问题建模为受限的多目标优化问题,将非支配排序遗传
算法(NSGA-II)和协方差矩阵适应进化策略(CMA-ES)应用于布局问题的求解,构建布局实现框架,
解决了高硬核利用率设计的布局困难问题,消除了手工布局硬核的麻烦,
且得到了超越退火和其他最先进布局算法的性能与5--6倍相对于Vivado EDA工作流的加速。
本文的主要工作分为以下几点:

\begin{enumerate}
    \item 回顾和总结了经典的脉动阵列加速器设计,指出加速器计算密度增大和高时钟频率的要求对EDA中的布局问题的挑战;
    \item 总结已有的FPGA布局算法并比较优劣,回顾已有对进化算法布局的尝试;
    \item 将异构FPGA的硬核布局建模为多目标优化问题,应用NSGA-II与CMA-ES两种进化算法求解。针对脉动阵列的级联规则造成的布局限制,提出一种创新的基因型设计,将布局问题划分为三个子问题并将布局限制
    编码进优化过程,避免产生不合法布局结果;    
    \item 量化布局运行时间、线长(wirelength)、卷积单元限制框尺寸、时钟频率等结果质量指标,并与标准布局工具VPR和目最先进的解析布局算法UTPlaceF对比,评价进化算法的性能;
    \item 设计和实现了自动化、端到端的快速FPGA布局布线框架,RapidLayout,并与Vivado手工布局工作流对比,评价实现效率的提升;
    \item 为硬核布局结果跨设备优化过程加速设计了布局的迁移学习方法,将布局方案迁移到UltraScale+ VU3P到VU13P的全系列FPGA,评价加速指标。
\end{enumerate}




\section{本文的论文结构与章节安排}
\label{sec:arrangement}
本文共分为五章,各章节内容安排如下:

第一章为引言,阐述了基于进化算法的FPGA脉动阵列布局方法选题的研究背景和意义,对国内外研究现状和相关工作的概述,
以及本文的主要创新点及研究内容,最终介绍本文的章节安排。


第二章是相关技术方法概述,简要介绍了基于进化算法的FPGA脉动阵列布局所涉及的各方面技术。首先总结了经典的
神经网络硬件加速器设计,然后介绍了FPGA的组成元件、EDA实现过程,并介绍了RapidWright FPGA 定制化
实现框架,最终总结了四类FPGA布局方法。

第三章是本文的重点内容,首先明确布局问题的范围并建模为多目标优化问题。其次,详细阐述了基于进化算法硬核布局
的基因型设计,并明确了NSGA-II与CMA-ES的实现原理。在问题建立的基础上,本文重点介绍了RapidLayout FPGA框架
的实现细节,最后层层递进地以VU11P FPGA的布局实现为例说明了RapidLayout的工作流程。

第四章是实验分析与结论,首先介绍了实验中使用的FPGA器件和实验中涉及的对比布局算法,其次描述了实验的编程实现与
测试环境等实现细节,然后以多种方法呈现布局方法性能与收敛过程的对比结果并加以分析,最后展示了RapidLayout的迁移
学习性能,说明了RapidLayout对其他FPGA的适用性。

第五章是总结与展望,总结了本文的内容,分析所提出方法与框架的优点与不足,然后进行未来展望,说明此项工作
未来的发展方向。

