\chapter{引言}
% 引言是论文正文的开端,应包括毕业论文选题的背景、目的和意义;对国内外研究现状和相关领域中已有的研究成果的简要评述;介绍本项研究工作研究设想、研究方法或实验设计、理论依据或实验基础;涉及范围和预期结果等。要求言简意赅,注意不要与摘要雷同或成为摘要的注解。

本章首先介绍基于FPGA的神经网络脉动阵列加速器的应用背景,大规模高计算密度的设计对FPGA实现过程带来的挑战以及FPGA布局的选题背景,然后总结了
已有的FPGA布局方法和研究现状,接着介绍本文的主要工作,最后概述本文的结构与章节安排。

\label{cha:introduction}
\section{选题背景与意义}
\label{sec:background}

% 脉动阵列加速器

% 现代异构FPGA 
现代的高性能FPGA在保留其灵活性的同时可以实现近似ASIC的计算性能与时钟频率,这得益于其异构计算资源的设计:除了数百万传统的查找表和寄存器,现代异构
FPGA还在片上引入了DSP,RAM等硬核实现近存储计算,这样的硬核具有计算效率高、存储容量大、时钟频率高、支持级联等特点。同时,对可级联的硬核之间
设计了专用高速布线资源,可构建高速数据通道方便数据搬运。

% 布局方法

% 

\section{国内外研究现状和相关工作}
\label{sec:related_work}
近年来,……;


\subsection{卷积神经网络加速器}
卷积神经网络加速器发展介绍,几种经典的架构,其中包括脉动阵列

\subsection{FPGA布局方法}
几种经典的FPGA布局方法介绍,近年FPGA布局的研究发展热点

\subsection{进化算法}
进化算法的应用和简单介绍


\section{本文的研究内容与主要工作}
几点贡献

\section{本文的论文结构与章节安排}
\label{sec:arrangement}
本文共分为五章,各章节内容安排如下:

