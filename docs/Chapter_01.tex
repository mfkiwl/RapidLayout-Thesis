\chapter{引言}
% 引言是论文正文的开端,应包括毕业论文选题的背景、目的和意义;对国内外研究现状和相关领域中已有的研究成果的简要评述;介绍本项研究工作研究设想、研究方法或实验设计、理论依据或实验基础;涉及范围和预期结果等。要求言简意赅,注意不要与摘要雷同或成为摘要的注解。

本章首先介绍基于FPGA的神经网络脉动阵列加速器的应用背景,大规模高计算密度的设计对FPGA实现过程带来的挑战以及FPGA布局的选题背景,然后总结了
已有的FPGA布局方法和研究现状,接着介绍本文的主要工作,最后概述本文的结构与章节安排。

\section{选题背景与意义}

% 脉动阵列加速器
英特尔的创始人之一戈登$\cdot$摩尔提出,每隔18-24个月集成电路可容纳的元器件就会翻一倍,其性能也将提升一倍,这条著名论断被人们称为摩尔定律(Moore's Law)。
然而随着晶体管逐渐变小,其静态功耗不断增加,功耗上升和能量效率降低使得摩尔定律逐渐失效。在“后摩尔时代”,研究者和从业者们不再单纯
追求计算密度的提高,转而寻求计算架构的创新。相对于CPU的复杂、低并行度,与GPU的通用、高并行度,FPGA为芯片设计者们提供了一种
领域定制架构(Domain-Specific Architecture)的全新思路:普通FPGA虽然频率较低,但具有高并行、可配置、低功耗、高能量效率的特点。
对比CPU集中和复杂的内存管理方式,FPGA的存储为分布式而且简单可配置。同时,现代FPGA多配备大量专用计算单元,
可提供较大规模的计算并行度。
现代的高性能FPGA在保留其灵活性的同时可以实现近似ASIC的计算性能与时钟频率,这得益于其异构计算资源的设计:除了数百万传统的查找表(LUTs)和寄存器(Flip-Flops),异构
FPGA还在片上引入了DSP,RAM等硬核IP实现近存储计算(Near-Storage Computation),这样的硬核具有计算效率高、存储容量大、时钟频率高、支持级联等特点。同时,对可级联的硬核之间
设计了专用高速布线资源,可构建高速数据通道方便数据搬运。然而,数量庞大的计算资源、分布和尺寸不规则的异构计算存储IP、逐渐增大的计算架构规模,都为
FPGA设计的实现带来了挑战。
以Xilinx Vivado Design Suite 在 UltraScale 与 UltraScale+ 系列FPGA上实现卷积神经网络加速器为例,完整的
实现流程需要耗费数小时甚至数天,并且无法保证实现结果可以达到目标时序要求。因此,面向大型异构高计算密度FPGA的布局布线算法成为了近年来
许多学者的研究重点。

% 布局方法

% 

\section{国内外研究现状和相关工作}
\label{sec:related_work}
近年来,……;


\subsection{卷积神经网络加速器}
卷积神经网络加速器发展介绍,几种经典的架构,其中包括脉动阵列

\subsection{FPGA布局方法}
几种经典的FPGA布局方法介绍,近年FPGA布局的研究发展热点

\subsection{进化算法}
进化算法的应用和简单介绍


\section{本文的研究内容与主要工作}
几点贡献

\section{本文的论文结构与章节安排}
\label{sec:arrangement}
本文共分为五章,各章节内容安排如下:

