\chapter{致谢}

% 由衷感谢我的导师某某教授,本文是在他的指导下完成的。……

% (谢辞应以简短的文字对课题研究与论文撰写过程中曾直接给予帮助的人员(例如指导教师、答疑教师及其他人员)表示对自己的谢意,这不仅是一种礼貌,也是对他人劳动的尊重,是治学者应当遵循的学术规范。内容限一页。)



时光飞逝,岁月如梭,值此毕业之际我想曾经对指导我、帮助我、和鼓励我的人献上诚挚的谢意。

感谢我的父母在大学期间给与我的无限支持与帮助,你们是我坚强的后盾。

感谢陈翔老师对我的培养,从我大二进入您实验室的两年来一直对我的悉心指导和无私帮助。回忆起在您指导下
完成的第一个FPGA卷积加速器项目使我感慨良多。在两年的时间里,
您给了我许多学习和成长的机会,鼓励着我逐渐发现了自己的研究兴趣,找到了未来的方向。
感谢您一直以来未曾动摇地支持和鼓励我的追求,未来如果我取得学术上的任何成就,那是与您的付出密不可分。

感谢滑铁卢大学的 Nachiket Kapre 教授对我细致入微的指导和帮助,和对
RapidLayout项目的巨大付出。感谢WatCAG实验室的各位伙伴。

感谢清华大学的汪玉老师对我的支持和帮助,感谢您在我访问学习的期间给了我很多鼓励,也为我提供了
实验的计算资源。感谢NOVAUTO公司的各位,在我忙于论文的时候帮我分担工作和安排时间。
感谢NICS实验室的各位师兄师姐和同学,以及两位行政老师,给了我接纳关怀和
学术工作的指导。


感谢符顺师兄和陈志鸿师兄对我的指引,感谢I3C实验室里的各位学长学姐给我的帮助。

感谢我的朋友们,感谢杨嵩毅同学和修圣杰同学在我成长学习过程中的帮助和陪伴。

最后我想感谢四年来学院对我的培养,感谢学院各位老师对我学业上的指引与帮助,感谢一路陪伴着我的同学们。








\vskip 108pt
\begin{flushright}
	张年崧\makebox[1cm]{} \\
	\today
\end{flushright}

