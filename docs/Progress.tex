% 第一次进度报告
\firstsummary{
	\begin{adjustwidth}{2em}{2em}
		本阶段(2019.10-2019.11),文献调研工作基本完成。总结出经典的卷积计算单元设计主要为两类,一类是线缓冲器设计(Line-Buffer),另一类是
		脉动阵列(Systolic Array)。线缓冲器设计主要针对内存读取受限的场景,擅长数据重用;而脉动阵列主要针对于计算受限的场景,规模大、速度快、并行
		度高,因此本文选择脉动阵列作为大规模神经网络加速器设计的基本架构。另一方面调研了FPGA实现的过程以及常用算法,对已有的布局算法进行分类,主要
		为基于模拟退火的布局算法、Min-Cut算法、解析算法、以及进化类算法,并进一步总结了其原理与特性。
	\end{adjustwidth}
}
% 第2次进度报告
\secondsummary{
	\begin{adjustwidth}{2em}{2em}
		本阶段(2019.11-2020.01),使用DSP48、BRAM、URAM三种硬核级联完成了脉动阵列加速器的RTL设计和仿真,以布局条件为约束、
		线长和计算单元边界框尺寸为目标函数,将硬核布局问题建模为存在约束的多目标优化问题。完成了RapidWright三种硬核布局和Site布线
		接口。
	\end{adjustwidth}
}
% 第3次进度报告
\thirdsummary{
	\begin{adjustwidth}{2em}{2em}
		本阶段(2020.01-2020.03),设计创新的基因型表示方法将布局可行解和限制条件编码,并应用非支配排序遗传算法与协方差矩阵改进进化策略
		对目标函数优化,基本完成了布局算法设计和测试。使用RapidWright与Vivado结合完成了逻辑网表复制、物理布局、Site布线、流水线寄存器插入
		、和SLR复制等功能,并整合成为完整的端到端布局布线框架。
	\end{adjustwidth}
}
% 第4次进度报告
\fourthsummary{
	\begin{adjustwidth}{2em}{2em}
		本阶段(2020.03-2020.04),设计布局结果可视化框架,将布局过程可视化为动态图像;
		对布局算法进行敏感度实验与参数优化搜索,确定最优参数组合,并且从最终时钟频率的角度验证布局结果的可靠性。经实验证明,
		本文提出的布局算法实现结果时钟频率均可超过URAM运行上限650 MHz,并进一步探究了流水线级数与时钟频率的关系;
		设计对比实验,与VPR、UTPlaceF等标准退火与解析布局工具对比运行时间和结果质量指标,重复实验得到结论,验证了本文提出的
		布局算法达到了领先标准。除此之外,完成了本文的撰写以及修改。
	\end{adjustwidth}
}



% 第1次老师评价
\firstcomment{
	\begin{adjustwidth}{2em}{2em}
		文献调研充分,研发工具学习准备完备,可按照计划展开相关研发工作。
	\end{adjustwidth}
}
% 第2次老师评价
\secondcomment{
	\begin{adjustwidth}{2em}{2em}
		已按照计划开展工作,进展顺利。
	\end{adjustwidth}
}
% 第3次老师评价
\thirdcomment{
	\begin{adjustwidth}{2em}{2em}
		已按照计划开展工作进展顺利,并且已完成相关IEEE国际会议论文投稿,工作完成度和质量很高。
	\end{adjustwidth}
}
% 第4次老师评价
\fourthcomment{
	\begin{adjustwidth}{2em}{2em}
		已按原定计划完成论文工作,可以送审。
	\end{adjustwidth}
}